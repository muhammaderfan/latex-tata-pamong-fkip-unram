%-----------------------------------------------------------------------------%
% Informasi Mengenai Dokumen
%-----------------------------------------------------------------------------%
% 
% Judul laporan. 
\var{\judul}{Pengembangan Science’s Live Worksheet (LKPD Sains Interaktif) Berorientasi Kemampuan Berpikir Tingkat Tinggi (HOTS) Siswa Kelas V Sekolah Dasar}
% 
% Tulis kembali judul laporan, kali ini akan diubah menjadi huruf kapital
\Var{\Judul}{Judul Skripsi/Thesis/Disertasi}
% 
% Tulis kembali judul laporan namun dengan bahasa Ingris
\var{\judulInggris}{Unknown Title for Final Report/Thesis/Disertation}

% 
% Tipe laporan, dapat berisi Skripsi, Tugas Akhir, Thesis, atau Disertasi
\var{\type}{Skripsi}
% 
% Tulis kembali tipe laporan, kali ini akan diubah menjadi huruf kapital
\Var{\Type}{Skripsi}
% 
% Tulis nama penulis 
\var{\penulis}{Nama Penulis}
% 
% Tulis kembali nama penulis, kali ini akan diubah menjadi huruf kapital
\Var{\Penulis}{Nama Penulis}
% 
% Tulis NPM penulis
\var{\npm}{NPM}
% 
% Tuliskan Fakultas dimana penulis berada
\Var{\Fakultas}{Ilmu Komputer}
\var{\fakultas}{Ilmu Komputer}
% 
% Tuliskan Program Studi yang diambil penulis
\Var{\Program}{ILMU KOMPUTER}
\var{\program}{Ilmu Komputer}
% 
% Tuliskan tahun publikasi laporan
\Var{\bulan}{Juli}
\Var{\tahun}{2019}
% 
% Tuliskan gelar yang akan diperoleh dengan menyerahkan laporan ini
\var{\gelar}{Sarjana Ilmu Komputer}
% 
% Tuliskan tanggal pengesahan laporan, waktu dimana laporan diserahkan ke 
% penguji/sekretariat
\var{\tanggalPengesahan}{XX Juli 2019} 
% 
% Tuliskan tanggal keputusan sidang dikeluarkan dan penulis dinyatakan 
% lulus/tidak lulus
\var{\tanggalLulus}{XX Juli 2019}
% 
% Tuliskan pembimbing 
\var{\pembimbing}{Prof. ???}
% 
% Alias untuk memudahkan alur penulisan paa saat menulis laporan
\var{\saya}{Muhammad Erfan, S.Pd., M.Pd.}
\var{\angSatu}{Anggota 1}
\var{\angDua}{Anggota 2}
\var{\angTiga}{Anggota 3}
\var{\angEmpat}{Anggota 4}
\var{\angLima}{Anggota 5}
\var{\mhsSatu}{Mahasiswa 1}
\var{\nimSatu}{NIM}
\var{\mhsDua}{Mahasiswa 1}
\var{\nimDua}{NIM}

%-----------------------------------------------------------------------------%
% Isyu Strategis
%----------------------------------------------------------------------------
\var{\isyu}{Membelajarkan kemampuan berpikir tingkat tinggi (HOTS) menggunakan LKPD Sains Interaktif}
\var{\topik}{Penelitian ini lebih berfokus pada pengembangan LKPD Sains Interaktif atau lembar kerja peserta didik (LKPD) interaktif pada materi IPA sekolah dasar yang berorientasi pada penguatan kemampuan berpikir tingkat tinggi (HOTS) yang dalam taksonomi Bloom termasuk dalam ranah kognitif C4, C5 dan C6. Pada LKPD Sains Interaktif tersebut siswa difasilitasi untuk menganalisis (C4) gejala-gejala alam di lingkungan, mengevaluasi (C5) inferensi terhadap berbagai fenomena alam yang ada di lingkungan sekitarnya, dan merancang (C6) kegiatan percobaan yang membuktikan suatu hipotesis}
\var{\objek}{Objek Penelitian adalah siswa kelas V (lima) SDN 6 Mataram}
\var{\lokasi}{Penelitian akan dilaksanakan di SDN 6 Mataram}
\var{\target}{Hasil yang ditargetkan dalam penelitian ini adalah dokumen (cetak maupun online) LKPD Sains Interaktif untuk siswa kelas V Sekolah Dasar yang berorientasi kemampuan berpikir tingkat tinggi (HOTS)}
%masukkan angka saja
\var{\biaya}{10.000.000}
\var{\instansi}{Universitas Mataram}
\var{\instansiLain}{Instansi lain yang terlibat dalam penelitian ini adalah SDN 6 Mataram}
\var{\temuan}{Karena Penelitian ini termasuk penelitian pengembangan, maka temuan yang ditargetkan dalam penelitian ini adalah deskripsi validitas, keterterapan, dan efektivitas LKPD Sains Interaktif siswa kelas V (lima) SDN 6 Mataram}
\var{\kontribusi}{Dokumen LKPD Sains Interaktif selain dapat digunakan oleh guru maupun siswa dalam pembelajaran di SDN 6 Mataram, dokumen LKPD Interaktif juga dapat digunakan oleh siapapun yang mengakses LKPD Interaktif secara daring. Selain dapat memberikan suatu pengalaman baru bagi guru, kemudahan yang ditawarkan LKPD Interaktif adalah dapat terintegrasi dengan berbagai platform pembelajaran yang tidak hanya dapat diakses melalui Laptop/PC guru atau siswa tetapi juga dapat diakses melalui smartphone}

\var{\jurnal}{Jurnal Pijar MIPA Universitas Mataram}
\var{\sinta}{4}

%----- RIncian Biaya
\var{\GajiUpah}{2.750.000}
\var{\HabisPakai}{5.900.000}
\var{\Perjalanan}{300.000}
\var{\Publikasi}{1.050.000}
\var{\TotalBiaya}{10.000.000}




%-----------------------------------------------------------------------------%
% Judul Setiap Bab
%-----------------------------------------------------------------------------%
% 
% Berikut ada judul-judul setiap bab. 
% Silahkan diubah sesuai dengan kebutuhan. 
% 
\Var{\kataPengantar}{Kata Pengantar}
\Var{\identitasUraian}{Identitas Dan Uraian Umum}
\Var{\babSatu}{Pendahuluan}
\Var{\babDua}{Tugas dan Fungsi}
\Var{\babTiga}{Mekanisme Standar Tata kelola}
\Var{\babEmpat}{BIAYA DAN JADWAL PENELITIAN}
